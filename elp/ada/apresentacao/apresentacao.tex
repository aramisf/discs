\documentclass{beamer}
\usepackage[brazil]{babel}
\usepackage[utf8]{inputenc}
%\usepackage{verbatim}

\title{Empacotamento e Tipagem Genérica em ADA}
\author{Aramis Fernandes \& Elder Crul}

\begin{document}

    % Uma das formas possiveis para declarar um frame:
    \frame{\titlepage}

    % [2]
    \frame
    {
        \frametitle{Sumário}
        \tableofcontents
    }

    % Manda os highlight da secao corrente:
    \AtBeginSection[]
    {
        \begin{frame}{Sumário}
            \tableofcontents[currentsection]
        \end{frame}
    }

    % Aqui a secao comeca efetivamente:
    \section{Empacotamento}

    % Cada frame eh um slide:
    \begin{frame}{O que é?}
        Segundo o Prof. Direne, é a Arte de ocultar gradualmente a informação.
    \end{frame}

    %%%
    \begin{frame}{E em ADA...}
        \begin{itemize}
            \item<1-> Existem dois tipos de pacotes:
            \begin{itemize}
                \item ADT (\textit{Abstract Data Type}): visíveis e privados;
                \item ASM (\textit{Abstract State Machine}): objeto.
            \end{itemize}
            \item<2-> ASM.public: operações e não atributos.
            \item<2-> ASM.private: *TODOS* os atributos.
        \end{itemize}
    \end{frame}

    %%% Exemplos de código:
    \begin{frame}[containsverbatim]

        \frametitle{Exemplo:}
        \begin{verbatim}
-- Definicao de um pacote onde especificacoes,
-- variaveis, tipos, constantes, etc.,
-- sao todos visiveis pelos programas que
-- usarao o pacote:

package Public_Only_Package is

    type Range_10 is range 1 .. 10;

end Public_Only_Package;
        \end{verbatim}
    \end{frame}

    \begin{frame}[containsverbatim]

        \frametitle{Exemplo:}
        \begin{verbatim}
-- Pacote privado:
package Not_That_Public_Package is

    type Private_Type is private;

private

   type Private_Type is array (1 .. 10) of Integer;

end Not_That_Public_Package;
        \end{verbatim}
    \end{frame}

    % Manda os highlight da secao corrente:
    \AtBeginSection[]
    {
        \begin{frame}{Sumário}
            \tableofcontents[currentsection]
        \end{frame}
    }
    % A secao seguinte:
    \section{Unidades Genéricas}
    \begin{frame}{Polimorfismo Paramétrico}
        \begin{itemize}
            \item<1-> A idéia de reutilização de código surge devido à
            necessidade de construir programas baseados em blocos de construção
            bem estabelecidos que possam ser combinados para formar um sistema
            mais amplo e complexo;
            \item<2-> A reutilização de código melhora a produtividade e a
            qualidade do \textit{software};
        \end{itemize}
    \end{frame}

    \begin{frame}{Polimorfismo Paramétrico}
        \begin{itemize}
            \item<1-> Uma das maneiras em que a linguagem \textbf{ADA} suporta
            essa característica é por meio de unidades de genéricos. Uma unidade
            genérica é uma unidade que define algoritmos em termos de tipos e
            operações que não estão definidas até o momento em que são
            instanciados;
            \item<2-> Eventualmente escrevemos código similar, com pequenas
            mudanças nos tipos das variáveis, mas com os mesmos algoritmos;
        \end{itemize}
    \end{frame}

    \begin{frame}[containsverbatim]
        \frametitle{Exemplo de Sobrecarga de tipo:}
        \begin{verbatim}
-- Na especificacao
generic
    type Element_T is private;

procedure Swap (X, Y : in out Element_T);

-- No corpo:
procedure Swap (X, Y : in out Element_T) is
    Temporary : constant Element_T := X;
begin
    X := Y;
    Y := Temporary;
end Swap;
        \end{verbatim}
    \end{frame}

    \begin{frame}[containsverbatim]
        \frametitle{Instanciando Entidades Genéricas}
        \begin{itemize}
            \item<1-> As entidades declaradas como genéricas não podem ser
            usadas diretamente, é necessário instanciá-las antes, por exemplo:
            \begin{verbatim}

procedure Swap_Integers is new Swap (Integer);

            \end{verbatim}
            \item<2-> A procedure genérica pode ser instanciada para todos os
            tipos desejados. Pode ser instanciada com menores diferentes, ou se
            o mesmo identificador for usado, cada declaração sobrecarrega a
            procedure.
            \begin{verbatim}

procedure Instance_Swap is new Swap (Float);
procedure Instance_Swap is new Swap (Day_T);
procedure Instance_Swap is new Swap (Element_T => Stack_T);

            \end{verbatim}
        \end{itemize}
    \end{frame}

    \begin{frame}[containsverbatim]
        \frametitle{Tipos de unidades Genéricas}
        \begin{verbatim}
generic
    ... formal parameters ...
procedure Proc (...);

generic
    ... formal parameters ...
procedure Func (...) return ...;

generic
    ... formal parameters ...
package Pack is
    ...
end Pack;
        \end{verbatim}
    \end{frame}

    \begin{frame}{Subprogramas como Parâmetro}
        \begin{itemize}
            \item<1-> Escrever uma função genérica que retorne o maior entre
            dois elementos, independente de tipo;

            \item<2-> Talvez o tipo instanciado não tenha o operador $>$ (em
            tipos definidos pelo usuário, por exemplo);

            \item<3-> Neste caso, como comparar os elementos?
        \end{itemize}
    \end{frame}

    \begin{frame}[containsverbatim]
        \frametitle{Subprogramas como Parâmetro (sobrecarga de código)}
        \begin{verbatim}
-- Spec:
generic
  type ValueType is private;

  with function Compare(L, R: ValueType) return Boolean;

  function Maximum_Generic(L, R: ValueType)\
    return ValueType;
        \end{verbatim}
    \end{frame}

    \begin{frame}[containsverbatim]
        \frametitle{Subprogramas como Parâmetro (sobrecarga de código)}
        \begin{verbatim}
-- Body:
function Maximum_Generic (L, R: ValueType)\
    return ValueType is
  begin
      if Compare(L,R) then
          return L;
      else
          return R;
      end if;
  end Maximum_Generic;
        \end{verbatim}
    \end{frame}

    \begin{frame}[containsverbatim]
        \frametitle{Instanciando Maximum\_Generic como comparador de números com
        ponto flutuante}
        \begin{verbatim}
function Max_Gen_Float is
  new Maximum_Generic (ValueType => Float, Compare => ">");

function Min_Gen_Float is
  new Maximum_Generic (ValueType => Float, Compare => "<");
        \end{verbatim}
    \end{frame}

    \begin{frame}[containsverbatim]
        \frametitle{Instanciando a função Maximum\_Generic como "Car"}
        \begin{verbatim}
-- Definindo o tipo "carro"
type Car is record
  Id: Long_Long_Integer;
  NumRodas: Positive range 1..10;
  Cor: Color;
  HPs: Floar range 0.0 .. 2_000.0;
end record;

-- Definindo uma função que compara dois carros entre si:
function CarCompare (A, B: Car) return Boolean is
  begin
    if A.HPs > B.HPs then
        return True;
    else
        return False;
    end if;
  end CarCompare;
        \end{verbatim}
    \end{frame}

    \begin{frame}[containsverbatim]
        \frametitle{Instanciando a função Maximum\_Generic como "Car"}
        \begin{verbatim}
function MaxCar is
  new Maximum_Generic (ValueType => Car, Compare => CarCompare);

  Palio: Car := (1, 4, Blue, 65);
  BMW: Car := (1, 4, Blue, 260);

  MaxCars (Palio, BMW);
        \end{verbatim}
    \end{frame}

\end{document}
