\documentclass[a4paper,11pt]{article}
\usepackage[top=2.8cm,right=2.5cm,bottom=2.5cm,left=3cm]{geometry}
\usepackage[brazil]{babel}
\usepackage[utf8]{inputenc}
\usepackage{indentfirst}
%\usepackage{here}

% Bug porco do Diabo, pra tirar a data
\date{ }


%----------------------------------------------------------
% Inicio do documento
\begin{document}

\title{
Redes 2\\
Prova 2 - $1^o$ Semestre 2011
}

\author {
Prof. Elias P. Duarte Jr.
}

% Conforme a relevancia, comentar
\maketitle

\begin{enumerate}

\item As aplicações de tempo real, tais como jogos pela Internet e VoIP, que
possuem restrições temporais específicas sobre entrega de pacotes, não podem ser
construídas com TCP/IP. Por que?

\item Um dos algoritmos de controle de congestionamento para o TCP utiliza a
retransmissão rápida (\textit{fast retransmission}). \textbf{(A)} explique a
estratégia. \textbf{(B)} por que ela torna a retransmissão rápida?

\item É muito importante que o TCP calcule o intervalo de time-out de forma
precisa. O uso do desvio médio do RTT se mostrou melhor para estimar o time-out
do que simplesmente o cálculo da média de RTT's. Por que?

\item Quando se programa um servidor TCP, iterativo, ou concorrente, deve-se
definir o tamanho da fila de espera de clientes. Um tamanho sugerido para esta
fila é 5. Como você explica que um número tão pequeno seja usado mesmo para
servidores concorrentes que atendem centenas/milhares de clientes
simultaneamente?

\item O que imprime o programa abaixo? Dica: conte com cuidado quantos processos
executam cada instrução!
\\

\texttt{i=1;\\
pid = fork();\\
printf("\%d\char`\\n",i);\\
if (pid) i++;\\
fork();\\
printf("\%d\char`\\n",i);\\
}

\item Explique (inclusive com um desenho) como funciona a consulta recursiva no
DNS. Leve em conta todos os possíveis componentes envolvidos na consulta,
inclusive o resolvedor, servidor de nomes local e servidor raiz.

\item Faça uma comparação dos protocolos DHCP e RARP. Mostre 1 característica em
que são idênticos e 3 características em que diferem.

\item Existem duas famílias de algoritmos de roteamento: Bellman-Ford e
Dijkstra. Responda e explique: estes algoritmos podem ser aplicados tanto para o
roteamento interno a um Sistema Autônomo quanto para o roteamento entre Sistemas
Autônomos?



\end{enumerate}

% Fim do documento
\end{document}
