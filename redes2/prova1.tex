\documentclass[a4paper,11pt]{article}
\usepackage[top=2.8cm,right=2.5cm,bottom=2.5cm,left=3cm]{geometry}
\usepackage[brazil]{babel}
\usepackage[utf8]{inputenc}
\usepackage{indentfirst}
%\usepackage{here}

% Sem data:
\date{ }


%----------------------------------------------------------
% Inicio do documento
\begin{document}

\title{
Redes 2\\
Prova 1 - $1^o$ Semestre 2011
}

\author {
Prof. Elias P. Duarte Jr.
}

\maketitle

\begin{enumerate}

\item As camadas 5 e 6 do modelo OSI (Sessão e Apresentação) não estão presentes
no modelo TCP/IP da Internet. Responda: \textbf{(A)} O que aconteceu com a
funcionalidade destas camadas no modelo TCP/IP? \textbf{(B)} Existe alguma
vantagem em implementar estas funções em camadas exclusivas, como o modelo
ISO/OSI?

\item Os endereços IP são organizados em classes, sendo a classe D de
\textit{multicast}. \textbf{(A)} O que e multicast? \textbf{(B)} Cite 1 (uma)
situação em que o multicast melhora a eficiência da transmissão na Internet.

\item Considere o endereço IP \textbf{192.34.82.201} e a máscara de subrede
\textbf{255.255.255.192}. Responda: \textbf{(A)} qual host está sendo
endereçado?  \textbf{(B)} Quantas subredes internas a organização possui?

\item Com relação à fragmentação no protocolo IPv4 responda: \textbf{(A)} os
fragmentos podem ser re-fragmentados antes de chegar ao destino? \textbf{(B)} os
fragmentos podem ser juntados no pacote original antes de chegar ao destino? Por
que?

\item O campo TTL do pacote IPv4 permite determinar um tempo máximo de vida de
um pacote IP, antes que seja removido da rede. Explique como este campo é
implementado.

\item O CIDR permitiu que o IPv4 continuasse sendo usado na Internet por ter
resolvido \textit{dois} problemas. Explique.

\item Considere o código formado por mensagens de 3 bits, e checksum
correspondente a 1 único bit de paridade. \textbf{(A)} Mostre todas as
palavras deste código. \textbf{(B)} Qual a distância do código? Explique sua
resposta.

\item O popular comando \texttt{ping}, implementado em virtualmente todos os
sistemas operacionais, é uma implementação do ICMP Echo-Request-Reply.
Exatamente o que significa uma expressão bem sucedida do \texttt{ping}?

\end{enumerate}

\end{document}
