% ------------------------------------------------------------------------
% ------------------------------------------------------------------------
% abnTeX2: Modelo de Artigo Acadêmico em conformidade com
% ABNT NBR 6022:2003: Informação e documentação - Artigo em publicação
% periódica científica impressa - Apresentação
% ------------------------------------------------------------------------
% ------------------------------------------------------------------------

\documentclass[
	% -- opções da classe memoir --
	article,			% indica que é um artigo acadêmico
	11pt,				% tamanho da fonte
	oneside,			% para impressão apenas no verso. Oposto a twoside
	a4paper,			% tamanho do papel.
	% -- opções da classe abntex2 --
	%chapter=TITLE,		% títulos de capítulos convertidos em letras maiúsculas
	%section=TITLE,		% títulos de seções convertidos em letras maiúsculas
	%subsection=TITLE,	% títulos de subseções convertidos em letras maiúsculas
	%subsubsection=TITLE % títulos de subsubseções convertidos em letras maiúsculas
	% -- opções do pacote babel --
	english,			% idioma adicional para hifenização
	brazil,				% o último idioma é o principal do documento
	sumario=tradicional
	]{abntex2}


% ---
% PACOTES
% ---

% ---
% Pacotes fundamentais
% ---
\usepackage{lmodern}			% Usa a fonte Latin Modern
\usepackage[T1]{fontenc}		% Selecao de codigos de fonte.
\usepackage[utf8]{inputenc}		% Codificacao do documento (conversão automática dos acentos)
\usepackage{indentfirst}		% Indenta o primeiro parágrafo de cada seção.
\usepackage{nomencl} 			% Lista de simbolos
\usepackage{color}				% Controle das cores
\usepackage{graphicx}			% Inclusão de gráficos
\usepackage{microtype} 			% para melhorias de justificação
\usepackage{amsfonts}
% ---

% ---
% Pacotes adicionais, usados apenas no âmbito do Modelo Canônico do abnteX2
% ---
\usepackage{lipsum}				% para geração de dummy text
% ---

% ---
% Pacotes de citações
% ---
\usepackage[brazilian,hyperpageref]{backref}	 % Paginas com as citações na bibl
\usepackage[alf]{abntex2cite}	% Citações padrão ABNT
% ---

% ---
% Configurações do pacote backref
% Usado sem a opção hyperpageref de backref
\renewcommand{\backrefpagesname}{Citado na(s) página(s):~}
% Texto padrão antes do número das páginas
\renewcommand{\backref}{}
% Define os textos da citação
\renewcommand*{\backrefalt}[4]{
	\ifcase #1 %
		Nenhuma citação no texto.%
	\or
		Citado na página #2.%
	\else
		Citado #1 vezes nas páginas #2.%
	\fi}%
% ---

% ---
% Informações de dados para CAPA e FOLHA DE ROSTO
% ---
\titulo{Aritmética Modular e Teste de Primalidade}
\autor{Aramis S. H. Fernandes}
\local{Curitiba}
\data{2014}
% ---

% ---
% Configurações de aparência do PDF final

% alterando o aspecto da cor azul
\definecolor{blue}{RGB}{41,5,195}

% informações do PDF
\makeatletter
\hypersetup{
     	%pagebackref=true,
		pdftitle={\@title},
		pdfauthor={\@author},
    	pdfsubject={Modelo de artigo científico com abnTeX2},
	    pdfcreator={LaTeX with abnTeX2},
		pdfkeywords={abnt}{latex}{abntex}{abntex2}{atigo científico},
		colorlinks=true,       		% false: boxed links; true: colored links
    	linkcolor=blue,          	% color of internal links
    	citecolor=blue,        		% color of links to bibliography
    	filecolor=magenta,      		% color of file links
		urlcolor=blue,
		bookmarksdepth=4
}
\makeatother
% ---

% ---
% compila o indice
% ---
\makeindex
% ---

% ---
% Altera as margens padrões
% ---
\setlrmarginsandblock{3cm}{3cm}{*}
\setulmarginsandblock{3cm}{3cm}{*}
\checkandfixthelayout
% ---

% ---
% Espaçamentos entre linhas e parágrafos
% ---

% O tamanho do parágrafo é dado por:
\setlength{\parindent}{1.3cm}

% Controle do espaçamento entre um parágrafo e outro:
\setlength{\parskip}{0.2cm}  % tente também \onelineskip

% Espaçamento simples
\SingleSpacing

\begin{document}

% Retira espaço extra obsoleto entre as frases.
\frenchspacing
%---
% página de titulo
\maketitle

% resumo em português
\begin{resumoumacoluna}
 Aritmética Modular nos permite trabalhar com um intervalo restrito de
 inteiros. Esta característica é muito útil em contextos onde existe um limite
 de tamanho na representação dos mesmos, como é o caso dos computadores.
 A aritmética modular também apresenta um papel importante para a realização
 de testes de Primalidade, que até há pouco tempo atrás era um problema cuja
 solução não estava totalmente classificada, em termos de complexidade
 computacional.

 \vspace{\onelineskip}

\end{resumoumacoluna}


\section{Aritmética Modular}
\vspace{1.5em}

\subsection{O que é}

Aritmética Modular é um sistema que nos permite manipular números inteiros de
forma que sempre estejam dispostos em um intervalo finito.
Define-se \textit{x módulo N} como o resto da divisão de $x$ por $N$. Se $r$ é
tal valor, então $0 \leq r < N$.

Podemos usar esta noção para definir uma relação de equivalência entre
números: $x$ e $y$ são ditos \textit{congruentes módulo N} se $x-y$ é múltiplo
de $N$:

$x \equiv y \pmod N \iff N$ divide $(x-y)$
%$x \equiv y \pmod N \Leftrightarrow N divide (x-y)$
\\

\textit{Exemplo:}
$53 \equiv 5 \pmod{12}$, pois $53-5 = 48$, que é múltiplo de 12. Em outras
palavras, 53 meses equivalem a 4 anos e 5 meses, pois 12 divide $(53-5)$.

%%%%%%%%%%%%%%%%%%%%%%%%%%%%%%%%%%%%%%%%%%%%%%%%%%%%%%%%%%%%%%%%%%%%%%%%%%
\vspace{1.5em}
\subsection{Classes de equivalência}

Utilizando artimética modular, podemos dividir os inteiros em intervalos cuja
forma segue um padrão:\\

\quad $\{i + kN \; | \; kN \in \mathbb{Z}\}$

Podemos chamar tais padrões de \textit{Classes de equivalência}, no caso, para
algum $i$ entre $0$ e $N-1$. Por exemplo, há duas classes de equivalência
módulo 3:\\

$\quad ... -9 -6 -3 0 3 6 9 ...$\\
$\quad ... -8 -5 -2 1 4 7 10 ...$\\
$\quad ... -7 -4 -1 2 5 8 11 ...$\\

Considerando cada uma das classes de equivalência, temos que os números que
delas participam são iguais, ou seja, 3 e 6 não apresentam qualquer diferença.

\subsection{Regra da Substituição Modular}
\subsection{Regra da Adição Modular}
\subsection{Regra da Multiplicação Modular}

\vspace{2.8em}
\section{Teste de Primalidade}

\subsection{Primeira Seção}

%\nocite{Dasgupta2009}
\vspace{2.8em}
\bibliographystyle{acm}
\bibliography{reflatex}
\end{document}
