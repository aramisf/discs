% ------------------------------------------------------------------------
% ------------------------------------------------------------------------
% abnTeX2: Modelo de Artigo Acadêmico em conformidade com
% ABNT NBR 6022:2003: Informação e documentação - Artigo em publicação
% periódica científica impressa - Apresentação
% ------------------------------------------------------------------------
% ------------------------------------------------------------------------

\documentclass[
	% -- opções da classe memoir --
	article,			% indica que é um artigo acadêmico
	11pt,				% tamanho da fonte
	oneside,			% para impressão apenas no verso. Oposto a twoside
	a4paper,			% tamanho do papel.
	% -- opções da classe abntex2 --
	%chapter=TITLE,		% títulos de capítulos convertidos em letras maiúsculas
	%section=TITLE,		% títulos de seções convertidos em letras maiúsculas
	%subsection=TITLE,	% títulos de subseções convertidos em letras maiúsculas
	%subsubsection=TITLE % títulos de subsubseções convertidos em letras maiúsculas
	% -- opções do pacote babel --
	english,			% idioma adicional para hifenização
	brazil,				% o último idioma é o principal do documento
	sumario=tradicional
	]{abntex2}


% ---
% PACOTES
% ---

% ---
% Pacotes fundamentais
% ---
\usepackage{lmodern}			% Usa a fonte Latin Modern
\usepackage[T1]{fontenc}		% Selecao de codigos de fonte.
\usepackage[utf8]{inputenc}		% Codificacao do documento (conversão automática dos acentos)
\usepackage{indentfirst}		% Indenta o primeiro parágrafo de cada seção.
\usepackage{nomencl} 			% Lista de simbolos
\usepackage{color}				% Controle das cores
\usepackage{graphicx}			% Inclusão de gráficos
\usepackage{microtype} 			% para melhorias de justificação
\usepackage{amsfonts}
% ---

% ---
% Pacotes adicionais, usados apenas no âmbito do Modelo Canônico do abnteX2
% ---
\usepackage{lipsum}				% para geração de dummy text
% ---

% ---
% Pacotes de citações
% ---
\usepackage[brazilian,hyperpageref]{backref}	 % Paginas com as citações na bibl
\usepackage[alf]{abntex2cite}	% Citações padrão ABNT
% ---

% ---
% Configurações do pacote backref
% Usado sem a opção hyperpageref de backref
\renewcommand{\backrefpagesname}{Citado na(s) página(s):~}
% Texto padrão antes do número das páginas
\renewcommand{\backref}{}
% Define os textos da citação
\renewcommand*{\backrefalt}[4]{
	\ifcase #1 %
		Nenhuma citação no texto.%
	\or
		Citado na página #2.%
	\else
		Citado #1 vezes nas páginas #2.%
	\fi}%
% ---

% ---
% Informações de dados para CAPA e FOLHA DE ROSTO
% ---
\titulo{Aritmética Modular e Teste de Primalidade}
\autor{Aramis S. H. Fernandes}
\local{Curitiba}
\data{2014}
% ---

% ---
% Configurações de aparência do PDF final

% alterando o aspecto da cor azul
\definecolor{blue}{RGB}{41,5,195}

% informações do PDF
\makeatletter
\hypersetup{
     	%pagebackref=true,
		pdftitle={\@title},
		pdfauthor={\@author},
    	pdfsubject={Modelo de artigo científico com abnTeX2},
	    pdfcreator={LaTeX with abnTeX2},
		pdfkeywords={abnt}{latex}{abntex}{abntex2}{atigo científico},
		colorlinks=true,       		% false: boxed links; true: colored links
    	linkcolor=blue,          	% color of internal links
    	citecolor=blue,        		% color of links to bibliography
    	filecolor=magenta,      		% color of file links
		urlcolor=blue,
		bookmarksdepth=4
}
\makeatother
% ---

% ---
% compila o indice
% ---
\makeindex
% ---

% ---
% Altera as margens padrões
% ---
\setlrmarginsandblock{3cm}{3cm}{*}
\setulmarginsandblock{3cm}{3cm}{*}
\checkandfixthelayout
% ---

% ---
% Espaçamentos entre linhas e parágrafos
% ---

% O tamanho do parágrafo é dado por:
\setlength{\parindent}{1.3cm}

% Controle do espaçamento entre um parágrafo e outro:
\setlength{\parskip}{0.2cm}  % tente também \onelineskip

% Espaçamento simples
\SingleSpacing

\begin{document}

% Retira espaço extra obsoleto entre as frases.
\frenchspacing
%---
% página de titulo
\maketitle

% resumo em português
\begin{resumoumacoluna}
 Aritmética Modular nos permite trabalhar com um intervalo restrito de
 inteiros. Esta característica é muito útil em contextos onde existe um limite
 de tamanho na representação dos mesmos, como é o caso dos computadores.
 A aritmética modular também apresenta um papel importante para a realização
 de testes de Primalidade, que até há pouco tempo atrás era um problema cuja
 solução não estava totalmente classificada, em termos de complexidade
 computacional.

 \vspace{\onelineskip}

\end{resumoumacoluna}


\section{Aritmética Modular}
\vspace{1.5em}

\subsection{O que é}

Aritmética Modular é um sistema que nos permite manipular números inteiros de
forma que sempre estejam dispostos em um intervalo finito.
Define-se \textit{x módulo N} como o resto da divisão de $x$ por $N$. Se $r$ é
tal valor, então $0 \leq r < N$.

Podemos usar esta noção para definir uma relação de equivalência entre
números: $x$ e $y$ são ditos \textit{congruentes módulo N} se $x-y$ é múltiplo
de $N$:

$x \equiv y \pmod N \iff N$ divide $(x-y)$
%$x \equiv y \pmod N \Leftrightarrow N divide (x-y)$
\\

\textit{Exemplo:}
$53 \equiv 5 \pmod{12}$, pois $53-5 = 48$, que é múltiplo de 12. Em outras
palavras, 53 meses equivalem a 4 anos e 5 meses, pois 12 divide $(53-5)$.

%%%%%%%%%%%%%%%%%%%%%%%%%%%%%%%%%%%%%%%%%%%%%%%%%%%%%%%%%%%%%%%%%%%%%%%%%%
\vspace{1.5em}
\subsection{Classes de equivalência}

Utilizando artimética modular, podemos dividir os inteiros em intervalos cuja
forma segue um padrão:\\

\quad $\{i + kN \; | \; kN \in \mathbb{Z}\}$

\vspace{2.8em}
\section*{Teste de Primalidade}

%%%%%%%%%%%%%%%%%%%%%%%%%%%%%%%%%%%%%%%%%%%%%%%%%%%%%%%%%%%%%%%%%%%%%%%%%%
\vspace{1.5em}
\subsection*{Fatoração}

O processo da fatoração é um processo bastante simples de ser compreendido,
contudo, apesar de existirem algumas otimizações possíveis, ele continua sendo
um procedimento computacionalmente custoso. A melhor otimização para este
procedimento é seguir o algoritmo de Eratóstenes, que consiste basicamente em
testar todos os números primos menores que a raiz quadrada do número a ser
testado.

O algoritmo começa com uma lista de todos os naturais até um dado número, que,
quando analisamos o caso otimizado, é $\sqrt{n}$. Testa se $n$ é divisível por 2, se não for,
elimina da lista todos os múltiplos de 2, e testa a divisão por 3. Se $n$ não é
divisível por 3, então elimina-se todos os múltiplos de 3 que ainda não foram
eliminados da lista. O algoritmo segue este método até atingir $\sqrt{n}$.
Caso $n$ não seja divisível por nenhum dos números testados, então pode-se
afirmar com certeza que $n$ é primo.

Apesar desse algoritmo ser caro em termos computacionais, ele retorna com a
resposta correta. Contudo, para casos práticos, onde a primalidade a ser
descoberta é a de números muito grandes, o processo da fatoração deixa de ser
interessante. Veremos a seguir um método mais eficaz, que apesar de ser
probabilístico, apresenta um custo computacional muito menor.


%%%%%%%%%%%%%%%%%%%%%%%%%%%%%%%%%%%%%%%%%%%%%%%%%%%%%%%%%%%%%%%%%%%%%%%%%%
\vspace{1.5em}
\subsection*{Teorema de Fermat}

Surgiu no ano 1640, e segundo ele é possível dizer se um número é primo sem
que seja necessário efetuar a fatoração (que é um processo custoso, como
vimos anteriormente).

Seu teorema diz o seguinte:

\textit{Se $p$ é primo, então para todo $1 \leq a < p$},

$\quad a^{p-1} \equiv 1 \pmod p$\\


\textit{Prova:} Seja $S = \{1,2,\ldots,p-1\}$. Provaremos que, quando todos os
elementos de $S$ são multiplicados por $a$ módulo $p$, os números resultantes
são todos distintos e não-nulos. Ainda, por estarem todos dentro do intervalo
$[1,p-1]$, necessariamente eles são uma permutação de $S$.

Os números $a\cdot i\bmod p$ são todos distintos porque se $a\cdot i \equiv
a\cdot j \pmod p$, então basta dividir ambos por $a$ para obtermos $i\equiv j
\pmod p$.

Tais números também são não-nulos porque $a\cdot i \equiv 0$ significa que
$i\equiv 0$, pois assumimos que $a \ne 0$, portanto podemos dividir ambos os
lados por $a$.

Podemos então escrever o conjunto $S$ de duas maneiras:

$S = \{1,2,\ldots,p-1\} = \{a\cdot1 \bmod p, a\cdot2\bmod p,\ldots,
a\cdot(p-1)\bmod p\}$

Multiplicando os elementos em cada uma dessas representações, temos:

$(p-1)! \equiv a^{p-1} \cdot (p-1)! \pmod p$

Por termos assumido que $p$ é primo, $(p-1)$ é seu primo relativo, e portanto
podemos dividir ambos os lados por $(p-1)!$, obtendo:

$1 \equiv  a^{p-1} \pmod p$
\vspace{1.2em}

Este teorema nos dá a impressão de que nenhum teste de fatoração é necessário,
mas existe um detalhe sobre o teorema que é importante ressaltar: \textit{Todo
número primo passa no teste de Fermat, mas nem todo número composto reprova}.
Por exemplo, $341 = 11 \cdot 31$ não é primo e ainda assim $2^{340} \equiv 1
\bmod 341$.

Um número composto tem de falhar para algum valor de $a$, vejamos então como
encontrar os erros.

\textbf{Lema} \textit{Se $a^{N-1} \not\equiv 1 \bmod N$ para algum primo
relativo de $N$, então o mesmo acontece para pelo menos metade das escolhas
de $a < N$.}

\textit{Prova.} Fixe algum valor de $a$ para o qual $a^{N-1}\not\equiv 1 \bmod
N$. Qualquer elemento $b < N$ que passa no teste de Fermat com relação a $N$
(ou seja, $b^{N-1} \equiv 1 \bmod N$) possui um gêmeo, $a\cdot b$, que falha
no teste:

$(a\cdot b)^{N-1}\equiv a^{N-1}\cdot b^{N-1}\equiv a^{N-1} \not\equiv 1 \bmod
N$

Além disso, todos esses elementos $a\cdot b$, para um $a$ fixo, para diferentes
$b$'s são distintos pela mesma razão que $a\cdot i \not\equiv a\cdot j$ na
prova do teste de Fermat.

Podemos então afirmar:

Se $N$ é primo, então $a^{N-1} \equiv 1 \bmod N$ para todo $a<N$.\\
Se $N$ não é primo, então $a^{N-1} \equiv 1 \bmod N$ para pelo menos metade
dos valores de $a<N$.

O algoritmo do nosso teste, então, possui um comportamento probabilístico,
onde:

Se $N$ é primo, o algoritmo retorna 1 com probabilidade 1;\\
Se $N$ não é primo, o algoritmo retorna 1 com probabilidade $\leq \frac{1}{2}$

Este último caso ocorre uma vez para cada um dos $a<N$, se rodarmos o
algoritmo para cada uma das $N-1$ vezes, podemos reduzir a probabilidade de
erro para $\frac{1}{2^{N-1}}$. Ou seja, a probabilidade de erro cai
exponencialmente.

Existe ainda um conjunto raro de números que passam no teste de Fermat para
todo $a$ primo relativo de $N$, são os chamados números de
\textit{Carmichael}. Vamos falar brevemente sobre eles agora.


%%%%%%%%%%%%%%%%%%%%%%%%%%%%%%%%%%%%%%%%%%%%%%%%%%%%%%%%%%%%%%%%%%%%%%%%%%
\vspace{1.5em}
\subsection*{Números de \textit{Carmichael}}

São números que passam no teste de Fermat para todos os primos relativos de um
certo $N$. O menor número de \textit{Carmichael} é 561. Ele não é primo, no
entanto $a^{560} \equiv 1 \pmod 561$ para todos os valores de $a$ que são
primos relativos de 561.

Por serem números muito raros, por muito tempo se pensou que fossem finitos,
mas hoje sabe-se que os números de Carmichael são infinitos, porém,
raríssimos.

Para contornar o problema que eles geram, existe um teste de primalidade
ligeiramente mais refinado. Escreva $N-1$ na forma $2^tu$, como no algoritmo
já estudado, escolheremos aleatoriamente um valor $a$ e verificaremos $a^{N-1}
\bmod N$. Realizamos essa computação primeiro determinando $a^u$, e depois
faremos sucessivas elevações ao quadrado, afim de obter:

$a^u\bmod N, a^{2u}\bmod N, \ldots, a^{2^tu} \equiv a^{N-1}\bmod N$

Se $a^{N-1} \not\equiv 1 \bmod N$, quando $N$ é composto segundo o pequeno
teorema de Fermat, então terminamos. Mas se $a^{N-1} \equiv \bmod N$,
conduzimos mais um teste: em algum momento da sequência anterior, passamos por
1 pela primeira vez. Se isso aconteceu depois da primeira posição (ou seja, se
$a^u \bmod N \ne 1$), e se o valor anterior da lista não é $-1 \bmod N$,
declaramos $N$ composto.

Neste caso, encontramos uma \textit{raiz quadrada não trivial} de 1 módulo
$N$: um número que não é $\pm 1\bmod N$, mas que, quando elevado ao quadrado,
é igual a $1\bmod N$. Tal número somente pode existir se $N$ é composto.

Através dessa verificação da raiz quadrada, em conjunto com o teste anterior
de Fermat, pelo menos $\frac{3}{4}$ dos possíveis valores de $a$ entre
$1..N-1$ revelarão um número composto $N$, mesmo se ele é um número de
Carmichael.


%%%%%%%%%%%%%%%%%%%%%%%%%%%%%%%%%%%%%%%%%%%%%%%%%%%%%%%%%%%%%%%%%%%%%%%%%%
\vspace{1.5em}
\subsection*{Gerando primos aleatórios}

Nosso próximo objetivo é obter um algortimo rápido para selecionar números
primos aleatórios com algumas centenas de bits de tamanho. Segundo o teorema
dos números primos de Lagrange, os números primos são abundantes: um número
aleatório de $n$ bits de comprimento tem uma chance aproximada de
$\frac{1,44}{n}$ de ser primo.\\
\\
\textbf{Teorema dos números primos de Lagrange} \textit{Seja $\pi (x)$ o
número de primos $\leq x$. Então $\pi (x) \approx x/(\ln x)$, ou mais
precisamente,}

$\quad \lim_{x \to \infty}\frac{\pi(x)}{(x/\ln x)} = 1$

\vspace{1em}
Tal abundância facilita a geração de um primo aleatório de $n$ bits:
\begin{itemize}
  \item Selecione um número aleatório $N$ de $n$ bits;
  \item Execute um teste de primalidade sobre $N$;
  \item Se ele passar no teste, mostre-o; senão volte ao início.
\end{itemize}

Quão rápido é esse algoritmo? Se o $N$ selecionado aleatoriamente for
realmente primo, o que acontece com probabilidade de pelo menos $1/n$, ele
certamente passará no teste. Portanto, o algortimo vai parar em
aproximadamente $\mathcal{O}(n)$ rodadas.

Mas qual o melhor teste de primalidade a ser aplicado? Como em nosso caso
estamos utilizando números escolhidos aleatoriamente (e não por um
adversário), é suficiente realizar o teste de Fermat com base $a=2$ (mas para
ser mais eficaz, podemos utilizar $\{2,3,5\}$, porque para números aleatórios
o teste de Fermat tem uma probabilidade de falha muito menor que a taxa de
pior caso que vimos anteriormente, de $1/2$. O algortimo resultante é bastante
rápido, gerando primos que têm centenas de bits de comprimento em uma fração
de segundo, mesmo num computador comum.

Mas qual a probabilidade de um número apresentado pelo algoritmo ser de fato
um primo?

Para compreendermos a eficácia do teste de Fermat, vamos estudar o seguinte
cenário:

Suponha que realizemos um teste com base $a=2$ para todos os $N \leq 25\times
10^9$. Nesse intervalo há cerca de $10^9$ primos e cerva de 20.000 compostos
que passam no teste. Portanto a chance de gerar erradamente um composto é
aproximadamente $20.000/10^9 = 2\times 10^{-5}$. Mas essa chance de erro
decresce rapidamente, à medida que aumentamos o comprimento dos números
envolvidos.

%\nocite{Dasgupta2009}
\vspace{2.8em}
\bibliographystyle{acm}
\bibliography{reflatex}
\end{document}
