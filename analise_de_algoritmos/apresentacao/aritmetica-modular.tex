\section{Aritmética Modular}
\vspace{1.5em}

\subsection{O que é}

Aritmética Modular é um sistema que nos permite manipular números inteiros de
forma que sempre estejam dispostos em um intervalo finito.
Define-se \textit{x módulo N} como o resto da divisão de $x$ por $N$. Se $r$ é
tal valor, então $0 \leq r < N$.

Podemos usar esta noção para definir uma relação de equivalência entre
números: $x$ e $y$ são ditos \textit{congruentes módulo N} se $x-y$ é múltiplo
de $N$:

$x \equiv y \pmod N \iff N$ divide $(x-y)$
%$x \equiv y \pmod N \Leftrightarrow N divide (x-y)$
\\

\textit{Exemplo:}
$53 \equiv 5 \pmod{12}$, pois $53-5 = 48$, que é múltiplo de 12. Em outras
palavras, 53 meses equivalem a 4 anos e 5 meses, pois 12 divide $(53-5)$.

%%%%%%%%%%%%%%%%%%%%%%%%%%%%%%%%%%%%%%%%%%%%%%%%%%%%%%%%%%%%%%%%%%%%%%%%%%
\vspace{1.5em}
\subsection{Classes de equivalência}

Utilizando artimética modular, podemos dividir os inteiros em intervalos cuja
forma segue um padrão:\\

\quad $\{i + kN \; | \; kN \in \mathbb{Z}\}$

Podemos chamar tais padrões de \textit{Classes de equivalência}, no caso, para
algum $i$ entre $0$ e $N-1$. Por exemplo, há duas classes de equivalência
módulo 3:\\

$\quad ... -9 -6 -3 0 3 6 9 ...$\\
$\quad ... -8 -5 -2 1 4 7 10 ...$\\
$\quad ... -7 -4 -1 2 5 8 11 ...$\\

Considerando cada uma das classes de equivalência, temos que os números que
delas participam são iguais, ou seja, 3 e 6 não apresentam qualquer diferença.

\subsection{Regra da Substituição Modular}
\subsection{Regra da Adição Modular}
\subsection{Regra da Multiplicação Modular}
