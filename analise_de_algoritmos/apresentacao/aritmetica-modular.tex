\section{Aritmética Modular}
\vspace{1.5em}

\subsection{O que é}

Aritmética Modular é um sistema que nos permite manipular números inteiros de
forma que sempre estejam dispostos em um intervalo finito.
Define-se \textit{x módulo N} como o resto da divisão de $x$ por $N$. Se $r$ é
tal valor, então $0 \leq r < N$.

Podemos usar esta noção para definir uma relação de equivalência entre
números: $x$ e $y$ são ditos \textit{congruentes módulo N} se $x-y$ é múltiplo
de $N$:

$x \equiv y \pmod N \iff N$ divide $(x-y)$
%$x \equiv y \pmod N \Leftrightarrow N divide (x-y)$
\\

\textit{Exemplo:}
$53 \equiv 5 \pmod{12}$, pois $53-5 = 48$, que é múltiplo de 12. Em outras
palavras, 53 meses equivalem a 4 anos e 5 meses, pois 12 divide $(53-5)$.

%%%%%%%%%%%%%%%%%%%%%%%%%%%%%%%%%%%%%%%%%%%%%%%%%%%%%%%%%%%%%%%%%%%%%%%%%%
\vspace{1.5em}
\subsection{Classes de equivalência}

Utilizando artimética modular, podemos dividir os inteiros em intervalos cuja
forma segue um padrão:\\

\quad $\{i + kN \; | \; kN \in \mathbb{Z}\}$
