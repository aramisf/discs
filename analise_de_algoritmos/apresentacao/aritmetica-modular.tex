\section*{Aritmética Modular}

Aritmética Modular é um sistema que nos permite manipular números inteiros de
forma que sempre estejam dispostos em um intervalo finito.
Define-se \textit{x módulo N} como o resto da divisão de $x$ por $N$. Se $r$ é
tal valor, então $0 \leq r < N$.

Podemos usar esta noção para definir uma relação de equivalência entre
números: $x$ e $y$ são ditos \textit{congruentes módulo N} se $x-y$ é múltiplo
de $N$. Formalmente:

$x \equiv y \pmod N \iff N$ divide $(x-y)$\\

\textit{Exemplo:} $53 \equiv 5 \pmod{12}$, pois $53-5 = 48$, que é múltiplo de
12. Em outras palavras, 53 meses equivalem a 4 anos e 5 meses, pois 12 divide
$(53-5)$.


%%%%%%%%%%%%%%%%%%%%%%%%%%%%%%%%%%%%%%%%%%%%%%%%%%%%%%%%%%%%%%%%%%%%%%%%%%
\vspace{1.5em}
\subsection*{Classes de equivalência}

Utilizando artimética modular, podemos dividir os inteiros em intervalos cuja
forma segue um padrão:

\quad $\{i + kN \; | \; kN \in \mathbb{Z}\}$

Podemos chamar tais padrões de \textit{classes de equivalência}. Isso
significa que todos os números serão limitados a um intervalo pré-definido
(por $i$) entre $0$ e $N-1$. Por exemplo, há três classes de equivalência
módulo 3:

\quad ... -9 -6 -3 0 3 6 9 ...

\quad ... -8 -5 -2 1 4 7 10 ...

\quad ... -7 -4 -1 2 5 8 11 ...

Considerando cada uma delas, temos que os números contidos em cada classe são
iguais. Por exemplo, 3 e 6 não apresentam qualquer diferença, pois,
calculando seu valor em módulo 3, ambos são resultam em zero.


\subsection*{Regra da Substituição Modular}

Considere $x$ e $x'$, ambos \textit{congruentes módulo N}, e $y$ e
$y'$, também \textit{congruentes módulo N}. Graças ao fato de ambos
pertencerem à mesma classe de equivalência, podemos dizer que:

$x+y \equiv x'+y' \pmod N$ e $xy \equiv x'y' \pmod N$

pois

$x \equiv x' \pmod N$ e $y \equiv y' \pmod N$

\vspace{1em}
Continuam válidas as propriedades associativa, comutativa e distributiva entre
os operandos em aritmética modular:

$x+(y+z) \equiv (x+y)+z \pmod N$

$xy \equiv yx \pmod N$

$x(y+z) \equiv xy+xz \pmod N$

\vspace{1em}
Por exemplo, considere uma disciplina de $90h$ por semestre, onde cada aula
tem duração de $2h$. Se fosse necessário fazer um curso intensivo, e a
disciplina fosse ministrada continuamente, com início às $12h$ de uma
segunda-feira, ela terminaria às $(45 \times 2)\pmod{24}$ horas do último dia.
Como $(45 \equiv 21) \pmod{24}$, então, do último período de $24h$ seriam
ocupadas $21h$, e como o início foi às $12h$, então $21h$ equivalem às $9h$ da
manhã.

Aplicando-se a regra da substituição, pode-se perceber que é possível
simplificar bastante alguns cálculos aritméticos, reduzindo resultados
intermediários a seus restos módulo N. Veja um exemplo de como tais
simplificações podem ser de grande ajuda para cálculos grandes:

$2^{345} \equiv (2^5)^{69} \equiv 32^{69} \equiv 1^{69} \equiv 1 \pmod{31}$


%%%%%%%%%%%%%%%%%%%%%%%%%%%%%%%%%%%%%%%%%%%%%%%%%%%%%%%%%%%%%%%%%%%%%%%%%%
\vspace{1.5em}
\subsection*{Regra da Adição Modular}

Considere a adição regular de dois números $x$ e $y$ quaisquer. Como estamos
tratando de dois números módulo N, então a resposta se encontra no intervalo
$\{0..2(N-1)\}$. Mais precisamente, se a soma excede $N-1$, subtraímos N do total
para trazer o resultado novamente para o intervalo requerido.

Como você pode ver, o custo computacional desta operação é de uma soma e
eventualmente uma subtração, de número que nunca excedem $2N$. Portanto, o
custo é linear no tamanho dos números.

Usando notação assintótica, escrevemos o custo da Adição Modular como
$\mathcal{O}(n)$, sendo $n$ o tamanho da representação do número e $N$ o
número propriamente dito, ou seja, $n = \lceil\log{} N\rceil$


%%%%%%%%%%%%%%%%%%%%%%%%%%%%%%%%%%%%%%%%%%%%%%%%%%%%%%%%%%%%%%%%%%%%%%%%%%
\vspace{1.5em}
\subsection*{Regra da Multiplicação Modular}

Continuando com o mesmo raciocínio da seção anterior, podemos perceber que uma
multiplicação entre dois números quaisquer $x$ e $y$, ambos em módulo N, terá
como resultado um número dentro do intervalo $\{0..(N-1)^2\}$, que continua
tendo no máximo um comprimento igual a $2n$ bits:\\

$\log{(N-1)}^2 = 2\log{(N-1)} \leq 2n$\\

Para reduzir a resposta ao módulo N, computamos o resto dividindo-o por N,
usando o seguinte algoritmo (de tempo quadrático) para a divisão:

\begin{verbatim}
função divide (x,y)

Entrada: Dois inteiros de n bits x e y, onde y >= 1
Saída: O quociente e o resto de x dividido por y

se x=0:
  retorna (q,r) = (0,0)

(q,r) = divide(x/2,y)

q = 2q
r = 2r

se x é ímpar:
  r += 1

se r >= y:
  r -= y
  q += 1

retorna (q,r)
\end{verbatim}

A divisão modular é um pouco mais complicada que a divisão ordinária, no
entanto, sempre que ela é possível, pode ser resolvida em tempo
$\mathcal{O}(n)$. Mais adiante este tópico será tratado em detalhes.


%%%%%%%%%%%%%%%%%%%%%%%%%%%%%%%%%%%%%%%%%%%%%%%%%%%%%%%%%%%%%%%%%%%%%%%%%%
\vspace{1.5em}
\subsection*{Regra da Exponenciação Modular}

Suponha $x, y \textrm{ e } N$ com comprimento de várias centenas de bits. É
possível calcular $x^y \pmod N$ rapidamente?

Sabemos que o resultado será um número módulo N, e por isso ele próprio terá
um comprimento de N bits (o que significa algumas centenas de bits!). Contudo,
o valor de $x^y$ pode ser muitas vezes maior que isso.

Uma vez que estamos falando sobre aritmética modular, seria interessante
tentarmos aplicar a idéia de módulo às operações que estamos tentando fazer.
Uma sugestão inicial é calcular $x^y$ através de sucessivas multiplicações por
$x$ módulo $N$.

Porque estas multiplicações são sempre em módulo $N$, elas não tomam muito
tempo. Mas caso $y$ possua algumas centenas de bits de tamanho de
representação, então o custo total será exponencial com relação ao tamanho de
$y$. Ou seja, precisamos encontrar um meio menos custoso computacionalmente.

Por sorte, existe um meio!

Se ao invés de multiplicarmos por $x \textrm{ módulo } N$, elevarmos ao
quadrado a cada passo, então teremos o valor de $x^y$ calculado usando apenas
$\log y$ multiplicações:\\

$x \bmod N \to x^2 \bmod N \to x^4 \bmod N \to\cdots\to x^{2^{\lfloor\log
y\rfloor}} \bmod N$\\

Para determinarmos $x^y \bmod N$, basta multiplicar o subconjunto das
potências de 2 que compõem sua representação, em particular aqueles que fazem
parte da representação binária de $y$. Se $y$ fosse o número $13$, teria $4$
bits, e sua representação seria algo como:\\

$\quad x^{13} = x^{1101_2} = x^{1000_2} \cdot x^{100_2} \cdot x^{1_2}$

\vspace{1.2em}
Um algoritmo de tempo polinomial é possível.

\begin{verbatim}
função exponenciacao_modular (x, y, N)

Entrada: 2 número inteiros de n bits x e N, um expoente inteiro y
Saída: x**y mod N

se y = 0:
  retorna 1

z = exponenciacao_modular(x, y/2, N)

se y é par:
  retorna z*z mod N

senão:
  retorna x * (z*z mod N)

\end{verbatim}

Formulando a idéia do algoritmo recursivo acima, temos:
\[ x^y =
  \left \{
    \begin{array}{l l}
      (x^{\lfloor y/2 \rfloor})^2 & \quad \textrm{se $y$ é par}\\
      x\cdot(x^{\lfloor y/2 \rfloor})^2 & \quad \textrm{se $y$ é ímpar}
    \end{array}
  \right.
\]

Considerando $n$ como a quantidade de bits necessária para representar $x, y$
e $N$, o algoritmo vai parar após no máximo $n$ chamadas recursivas, e durante
cada chamada ele multiplica números de $n$ bits. Como a multiplicação é feita usando aritmética modular, 
seu custo é $\mathcal{O}(n^2)$, e como são feitas $n$ dessas multiplicações, o
custo total do algoritmo é de $\mathcal{O}(n^3)$.


%%%%%%%%%%%%%%%%%%%%%%%%%%%%%%%%%%%%%%%%%%%%%%%%%%%%%%%%%%%%%%%%%%%%%%%%%%
\vspace{1.5em}
\subsection*{Regra da Divisão Modular}


%%%%%%%%%%%%%%%%%%%%%%%%%%%%%%%%%%%%%%%%%%%%%%%%%%%%%%%%%%%%%%%%%%%%%%%%%%%
\vspace{1.3em}
\subsubsection*{Algoritmo de Euclides para o Máximo Divisor Comum}

Euclides foi um matemático grego, que viveu há mais de 2 mil anos atrás.
Dentre suas descobertas está um algoritmo que calcula o máximo divisor comum
entre dois inteiros $a$ e $b$.

A abordagem que aprendemos na escola, é fatorar ambos e multiplicar os
elementos comuns, encontrando assim o \textit{mdc}. Contudo, esta abordagem é
pouco eficiente.

O algoritmo de Euclides utiliza uma regra bastante simples.\\
\textbf{Regra de Euclides}: \textit{Se x e y são inteiros positivos com $x
\geq y$, então mdc(x,y) = mdc($x \bmod y, y$).}

A demonstração é simples e consiste basicamente na compreensão de uma regra
ainda mais simples que a de Euclides:\\

$\quad mdc(x,y) = mdc(x-y, y)$\\

Pode-se chegar a essa simplificação subtraindo-se $y$ de $x$ sucessivas vezes.

A prova segue diretamente: por um lado, se um inteiro divide $x$ e $y$, então este inteiro
também divide $x-y$, portanto $mdc(x,y) \leq mdc(x-y,y)$. Por outro lado, se
um inteiro divide tanto $x-y$ quanto $y$, então este inteiro também deve
dividir tanto $x$ quanto $y$, portanto $mdc(x,y) \geq mdc(x-y,y)$.

Segue um algoritmo recursivo que utiliza a regra de Euclides para calcular o
\textit{mdc}:

\begin{verbatim}
função Euclides(a,b)

Entrada: 2 inteiros a e b com a >= b >= 0
Saída: mdc(a,b)

se b = 0:
  retorna a

retorna Euclides(b, a mod b)

\end{verbatim}

Para calcular o tempo de execução deste algoritmo, é necessário entender a
velocidade de decrescimento dos argumentos a cada uma das chamadas recursivas.
Com uma chamada, os argumentos $(a,b)$ tornam-se $(b,a\bmod b)$: troca-se a
ordem e o tamanho de $a$ é reduzido substancialmente.

\textbf{Lema}: \textit{Se $a \geq b$, então $a \bmod b < a/2$.}

\textit{Prova}: Vamos dividir as possibilidades em dois casos distintos: ou $b
\leq a/2$ ou $b > a/2$.\\
Caso $b \leq a/2$, então $a \bmod b < b \leq a/2$.\\
Caso $b > a/2$, então $a \bmod b = a-b < a/2$.\\

Isso demonstra como, depois duas rodadas consecutivas quaisquer, tanto $a$
quanto $b$ foram reduzidos, no mínimo, pela metade de seu valor inicial - seu
tamanho decresce pelo menos 1 bit a cada rodada. Se ambos são inteiros com $n$
bits de tamanho, então o caso base será alcaçado em $2n$ rodadas. Assim como
temos visto ao longo desse estudo, nossa divisão modular possui tempo
quadrático, e como são efetuadas $2n$ chamadas, o tempo total do algoritmo é
$\mathcal{O}(n^3)$.


%%%%%%%%%%%%%%%%%%%%%%%%%%%%%%%%%%%%%%%%%%%%%%%%%%%%%%%%%%%%%%%%%%%%%%%%%%%
\vspace{1.3em}
\subsubsection*{Uma extensão do Algoritmo de Euclides}

Nesta seção faremos uma análise de uma extensão no algoritmo de Euclides que
nos permitirá calcular o \textit{mdc}.

\textbf{Lema} \textit{Se $d$ divide tanto $a$ quanto $b$ e $d = ax + by$ para
inteiros $x$ e $y$, então, necessariamente, $d = mdc(a,b)$}.

\textit{Prova}: Por um lado, considerando-se as duas primeiras condições, $d$
é divisor comum de $a$ e de $b$ e, portanto, não pode exceder o \textit{mdc},
ou seja $d \leq mdc(a,b)$. Por outro lado, como o $mdc(a,b)$ é comum a $a$ e a
$b$, ele deve necessariamente dividir $ax + by = d$, o que significa que
$mdc(a,b) \leq d$. Portanto, $d = mdc(a,b)$.

Os coeficientes $x$ e $y$ podem ser encontrados com uma extensão do algoritmo
de Euclides:

\begin{verbatim}
função euclides_estendido(a,b)

Entrada: 2 inteiros a e b com a >= b >= 0
Saída: Inteiros x, y e d tais que d = mdc(a,b) e ax + by = d

se b == 0:
  retorna (1,0,a)

(x',y',d) = euclides_estendido(b, a mod b)

retorna (y', x' - (a/b)*y, d)

\end{verbatim}

\textbf{Lema}: \textit{Para quaisquer inteiros positivos $a$ e $b$, o
algoritmo estendido de Euclides retorna inteiros $x, y e d$ tais que $mdc(a,b)
= d = ax + by$.}

\textit{Prova} Se forem ignorados $x$ e $y$, o algoritmo se comporta como o
original. Então, no mínimo, estaremos calculando $d = mdc(a,b)$.

A recursão termina quando $b=0$, portanto é interessante fazermos a indução no
valor de $b$. O caso base pode ser verificado diretamente, pois quando $b = 0$
o algoritmo retorna imediatamente o valor contido em $a$.

Para valores de $b$ maiores que zero, o algoritmo encontra o $mdc(a,b)$
chamando $mdc(b, a \bmod b)$. Como $a \bmod b < b$ podemos concluir, aplicando
a hipótese de indução e concluir que os $x'$ e $y'$ retornados estão
corretos:\\

$\quad mdc(b, a\bmod b) = bx' + (a \bmod b)y'$.\\

Escrevendo $(a \bmod b)$ como $(a - \lfloor a/b \rfloor b)$, encontramos:\\

$\quad d = mdc(a,b) = mdc(b, a\bmod b) = bx' + (a \bmod b)y' = bx' + (a -
\lfloor a/b \rfloor b)y' = ay' + b(x' - \lfloor a/b \rfloor y')$.\\

Desse modo, $d = ax + by$ com $x = y'$ e $y = x' - \lfloor a/b \rfloor y'$,
validando assim o comportamento do algoritmo sobre a entrada $(a,b)$.


%%%%%%%%%%%%%%%%%%%%%%%%%%%%%%%%%%%%%%%%%%%%%%%%%%%%%%%%%%%%%%%%%%%%%%%%%%%
\vspace{1.3em}
\subsubsection*{Divisão Modular}

Na aritmética comum, todo número $a \neq 0$ possui um inverso, $\frac{1}{a}$,
e dividir por $a$ é o mesmo que multiplicar pelo inverso. Dentro da aritmética
modular, uma definição similar é possível:\\

\textit{$x$ é o inverso multiplicativo de $a$, módulo N, se $ax \equiv 1 \pmod
N$.}\\

Pode haver no máximo um tal $x$ módulo N, e o denotaremos por $a^{-1}$ (ou
seja $aa^{-1} \equiv 1 \pmod N$).  Mas não é sempre que esse inverso
muliplicativo pode ser encontrado, pois ele nem sempre existe. Veja por
exemplo o número 2. Ele não é inversível módulo 6, ou seja, $\not\exists x |
2x \not\equiv 1 \bmod 6$.  Nesse caso, $a$ e $N$ são ambos pares e assim $a$
mod $N$ é sempre par, porque $a$ mod $N = a - kN$ para algum $k$.

Ainda, podemos ter certeza de que $mdc(a,N)$ divide $ax \bmod N$, pois esta
última expressão pode ser escrita na forma $ax + kN$. Assim, se $mdc(a,N) >
1$, então $ax \not\equiv 1 \bmod N$, não importa quem seja $x$ e, portanto,
$a$ não pode ter um inverso multiplicativo módulo $N$.

Este é o único caso onde $a$ não é inversível. Quando o $mdc(a,N) = 1$,
dizemos que $a$ e $N$ são \textit{primos relativos}. Quando temos dois números
que são primos relativos, o algoritmo estendido de Euclides nos dá inteiros
$x$ e $y$ tais que $ax + Ny = 1$, o que significa que $ax \equiv 1 \pmod N$,
ou seja, $x$ é o inverso de $a$.

\textbf{Teorema da Divisão Modular} \textit{Para qualquer $a$ mod $N$, a
possui um inverso multiplicativo módulo $N$ se e somente se ele é primo
relativo de $N$. Se existir, este inverso pode ser encontrado em tempo
$\mathcal{O}(n^3)$ executando o algoritmo estendido de Euclides.}

Assim fica resolvida a questão da Divisão Modular. Trabalhando com números
módulo $N$, podemos dividir por números primos relativos de $N$ (somente),
depois, para efetivar a divisão, basta multiplicar pelo seu inverso.

