\section*{Aritmética Modular}

Aritmética Modular é um sistema que nos permite manipular números inteiros de
forma que sempre estejam dispostos em um intervalo finito.
Define-se \textit{x módulo N} como o resto da divisão de $x$ por $N$. Se $r$ é
tal valor, então $0 \leq r < N$.

Podemos usar esta noção para definir uma relação de equivalência entre
números: $x$ e $y$ são ditos \textit{congruentes módulo N} se $x-y$ é múltiplo
de $N$. Formalmente:

$x \equiv y \pmod N \iff N$ divide $(x-y)$\\

\textit{Exemplo:} $53 \equiv 5 \pmod{12}$, pois $53-5 = 48$, que é múltiplo de
12. Em outras palavras, 53 meses equivalem a 4 anos e 5 meses, pois 12 divide
$(53-5)$.


%%%%%%%%%%%%%%%%%%%%%%%%%%%%%%%%%%%%%%%%%%%%%%%%%%%%%%%%%%%%%%%%%%%%%%%%%%
\vspace{1.5em}
\subsection*{Classes de equivalência}

Utilizando artimética modular, podemos dividir os inteiros em intervalos cuja
forma segue um padrão:

\quad $\{i + kN \; | \; kN \in \mathbb{Z}\}$

Podemos chamar tais padrões de \textit{classes de equivalência}. Isso
significa que todos os números serão limitados a um intervalo pré-definido
(por $i$) entre $0$ e $N-1$. Por exemplo, há três classes de equivalência
módulo 3:

\quad ... -9 -6 -3 0 3 6 9 ...

\quad ... -8 -5 -2 1 4 7 10 ...

\quad ... -7 -4 -1 2 5 8 11 ...

Considerando cada uma delas, temos que os números contidos em cada classe são
iguais. Por exemplo, 3 e 6 não apresentam qualquer diferença, pois,
calculando seu valor em módulo 3, ambos são resultam em zero.


\subsection*{Regra da Substituição Modular}

Considere $x$ e $x'$, ambos \textit{congruentes módulo N}, e $y$ e
$y'$, também \textit{congruentes módulo N}. Graças ao fato de ambos
pertencerem à mesma classe de equivalência, podemos dizer que:

$x+y \equiv x'+y' \pmod N$ e $xy \equiv x'y' \pmod N$

pois

$x \equiv x' \pmod N$ e $y \equiv y' \pmod N$

\vspace{1em}
Continuam válidas as propriedades associativa, comutativa e distributiva entre
os operandos em aritmética modular:

$x+(y+z) \equiv (x+y)+z \pmod N$

$xy \equiv yx \pmod N$

$x(y+z) \equiv xy+xz \pmod N$

\vspace{1em}
Por exemplo, considere uma disciplina de $90h$ por semestre, onde cada aula
tem duração de $2h$. Se fosse necessário fazer um curso intensivo, e a
disciplina fosse ministrada continuamente, com início às $12h$ de uma
segunda-feira, ela terminaria às $(45 \times 2)\pmod{24}$ horas do último dia.
Como $(45 \equiv 21) \pmod{24}$, então, do último período de $24h$ seriam
ocupadas $21h$, e como o início foi às $12h$, então $21h$ equivalem às $9h$ da
manhã.

Aplicando-se a regra da substituição, pode-se perceber que é possível
simplificar bastante alguns cálculos aritméticos, reduzindo resultados
intermediários a seus restos módulo N. Veja um exemplo de como tais
simplificações podem ser de grande ajuda para cálculos grandes:

$2^{345} \equiv (2^5)^{69} \equiv 32^{69} \equiv 1^{69} \equiv 1 \pmod{31}$


%%%%%%%%%%%%%%%%%%%%%%%%%%%%%%%%%%%%%%%%%%%%%%%%%%%%%%%%%%%%%%%%%%%%%%%%%%
\vspace{1.5em}
\subsection*{Regra da Adição Modular}

Considere a adição regular de dois números $x$ e $y$ quaisquer. Como estamos
tratando de dois números módulo N, então a resposta se encontra no intervalo
$\{0..2(N-1)\}$. Mais precisamente, se a soma excede $N-1$, subtraímos N do total
para trazer o resultado novamente para o intervalo requerido.

Como você pode ver, o custo computacional desta operação é de uma soma e
eventualmente uma subtração, de número que nunca excedem $2N$. Portanto, o
custo é linear no tamanho dos números.

Usando notação assintótica, escrevemos o custo da Adição Modular como
$\mathcal{O}(n)$, sendo $n$ o tamanho da representação do número e $N$ o
número propriamente dito, ou seja, $n = \lceil\log{} N\rceil$


%%%%%%%%%%%%%%%%%%%%%%%%%%%%%%%%%%%%%%%%%%%%%%%%%%%%%%%%%%%%%%%%%%%%%%%%%%
\vspace{1.5em}
\subsection*{Regra da Multiplicação Modular}

Continuando com o mesmo raciocínio da seção anterior, podemos perceber que uma
multiplicação entre dois números quaisquer $x$ e $y$, ambos em módulo N, terá
como resultado um número dentro do intervalo $\{0..(N-1)^2\}$, que continua
tendo no máximo um comprimento igual a $2n$ bits:\\

$\log{(N-1)}^2 = 2\log{(N-1)} \leq 2n$\\

Para reduzir a resposta ao módulo N, computamos o resto dividindo-o por N,
usando o seguinte algoritmo (de tempo quadrático) para a divisão:

\begin{verbatim}
função divide (x,y)

Entrada: Dois inteiros de n bits x e y, onde y >= 1
Saída: O quociente e o resto de x dividido por y

se x=0:
  retorna (q,r) = (0,0)

(q,r) = divide(x/2,y)

q = 2q
r = 2r

se x é ímpar:
  r += 1

se r >= y:
  r -= y
  q += 1

retorna (q,r)
\end{verbatim}

A divisão modular é um pouco mais complicada que a divisão ordinária, no
entanto, sempre que ela é possível, pode ser resolvida em tempo
$\mathcal{O}(n)$. Mais adiante este tópico será tratado em detalhes.


%%%%%%%%%%%%%%%%%%%%%%%%%%%%%%%%%%%%%%%%%%%%%%%%%%%%%%%%%%%%%%%%%%%%%%%%%%
\vspace{1.5em}
\subsection*{Regra da Exponenciação Modular}

Suponha $x, y \textrm{ e } N$ com comprimento de várias centenas de bits. É
possível calcular $x^y \pmod N$ rapidamente?

Sabemos que o resultado será um número módulo N, e por isso ele próprio terá
um comprimento de N bits (o que significa algumas centenas de bits!). Contudo,
o valor de $x^y$ pode ser muitas vezes maior que isso.

Uma vez que estamos falando sobre aritmética modular, seria interessante
tentarmos aplicar a idéia de módulo às operações que estamos tentando fazer.
Uma sugestão inicial é calcular $x^y$ através de sucessivas multiplicações por
$x$ módulo $N$.

Porque estas multiplicações são sempre em módulo $N$, elas não tomam muito
tempo. Mas caso $y$ possua algumas centenas de bits de tamanho de
representação, então o custo total será exponencial no tamanho de $y$. Ou
seja, precisamos encontrar um meio menos custoso computacionalmente.

Por sorte, existe um meio!

Se ao invés de multiplicarmos por $x \textrm{ módulo } N$, elevarmos ao quadrado a
cada passo, então teremos $x^y$ calculado usando apenas $\log y$
multiplicações:\\

$x \bmod N \to x^2 \bmod N \to x^4 \bmod N \to\cdots\to x^{2^{\lfloor\log
y\rfloor}} \bmod N$

%%%%%%%%%%%%%%%%%%%%%%%%%%%%%%%%%%%%%%%%%%%%%%%%%%%%%%%%%%%%%%%%%%%%%%%%%%
\vspace{1.5em}
\subsection*{Regra da Divisão Modular}

\vspace{1.3em}
\subsubsection*{Algoritmo de Euclides para o Máximo Divisor Comum}

\vspace{1.3em}
\subsubsection*{Uma extensão do Algoritmo de Euclides}

\vspace{1.3em}
\subsubsection*{Divisão Modular}
